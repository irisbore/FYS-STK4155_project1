 A significant amount of big data is geospatial data, and the size of such data is growing\cite{geospatital}. As big datasets become more common in geospatial analysis, it's important to assess how well standard linear regression methods work in these complex situations. Tests using synthetic data might not reveal the challenges posed by real-world data structures.  This study examines how three key linear regression methods perform when applied to different types of data: real-world high-dimensional topographic data and synthetic data generated from a uniform distribution. Our goal is to understand how these models behave under different conditions, such as adding constraints or changing the degrees of freedom. Our results show significant differences in model performance when applied to real-world data compared to synthetic data, especially concerning computational complexity and the influence of the data's structure. These findings highlight the difficulties of working with high-dimensional, real-world datasets and suggest that tailored approaches may be needed in these cases. \cite{openai2023chatgpt}