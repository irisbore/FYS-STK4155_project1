In statistics and machine learning, regression analysis is a fundamental tool for modeling and understanding relationships within data \cite{hastie2009elements}. Ordinary Least Squares (OLS), Ridge Regression, and Lasso Regression are pivotal techniques that address challenges such as overfitting and high dimensionality, each providing unique insights into data structures \cite{hoerl1970ridge}\cite{tibshirani1996lasso}.
\newline

High-dimensional, real-world datasets like topographic terrain data present significant challenges related to computational complexity and intricate data patterns \cite{bellman1961adaptive}. To evaluate model performance and study the bias-variance tradeoff in such contexts, resampling methods like Bootstrap and K-fold cross-validation are essential \cite{efron1993bootstrap}\cite{arlot2010survey}. These techniques ensure robust model evaluation and enhance the generalizability of results.
\newline

In this study, we apply OLS, Ridge, and Lasso regression methods to both real terrain data and uniformly distributed synthetic data. By comparing the Mean Squared Error (MSE) and R-squared (R²) metrics across these datasets, we aim to understand how each regression method performs under varying data conditions. We also utilize Bootstrap and K-fold cross-validation to examine the bias-variance tradeoff and to find the optimal regularization parameter lambda and polynomial degree for Ridge and Lasso regression.
\newline 

The report is structured as follows. We begin by introducing the data generation and preprocessing. Next, we describe the regression techniques and resampling methods used in this study. Further, we describe the experimental setup, detailing how the models were validated and evaluated. We then present and discuss our findings, focusing on model performance metrics and the reproduction of terrain data based on K-fold cross-validation. Finally, we summarize our conclusions and suggest directions for future research.
